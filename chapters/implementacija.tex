\chapter{Implementacija}
\label{chap:implementacija}

Aplikacija je pisana u programskom jeziku Python verzije 2.7. Aplikacija se
dijeli u nekoliko zasebnih cjelina. U središtu aplikacije nalazi se cjovovod
koji poziva alate poput BLAST-a i fastacmd-a za komuniciranje sa NCBI-jevom
neredundantnom bazom, zatim dio aplikacije za odabir i balansiranje vrsta na
taksonomskom stablu te alat mafft za poravnanje sekvenci.  Pored cjevovoda
implementirana je web aplikacija kao korisničko sučelje za cijeli program. Web
aplikacija je implementirana koristeći Flask microframework, dok su operacije na
klijentskoj strani implementirane u javascriptu uz korištenje biblioteke jQuery.


\section{Cjevovod}
\label{sec:cjevovod}

Cjevovod je arhitektonski programski obrazac u kojem prolaze kroz filtre koji su
postavljeni jedan za drugim. Time se simulira jedan tok koji ulazne podatke
transformacijom kroz filtre generira izlazne podatke. U ovome projektu
cjevovodna arhitektura je samo logički kostur koji se enkapsulira unutar razreda
\emph{Pipeline}. Iako je u začetku razvoja aplikacije svaki filter bio zaseban
proces, vrlo ubrzo je ustanovljeno kako većina filtera generira podatke koji su
potrebni na raznim mjestima u cijeloj aplikaciji te se činilo lakše imati sve
podatke u memoriji pojedinog cjevovoda. To je omogućilo da razred
\emph{Pipeline} naslijedi razred \emph{Thread} iz modula \emph{threading} te se
može pozivati kao zasebna dretva.

Tok cjevovoda se može vidjeti na slici \ref{fig:cjevovod}. Ulaz u cjevovod
predstavljaju paralogni proteini u FASTA formatu koje zadaje korisnik. Ti se
podaci zadaju pri stvaranju objekta \emph{Pipeline} kako bi se mogli zapisati na
disk u direktorij vezan za instancu \emph{Pipeline-a}. Stvarni objekt kojeg
prima konstruktor \emph{Pipeline-a} je riječnik prilagođen uporabi servera, što
je detaljnije opisano u sekciji \ref{sec:server}.

Pri pokretanju cjevovoda za svaku se od unešenih sekvenci stvara objetk razreda
\emph{ProteinHolder} prilikom čega se obavljaju pozivi filtara nezavisnih za
svaku pojedinu sekvencu. Prvi filter koji se koristi je alat \emph{BLAST} te je
izveden kao poziv zasebnog programa \emph{blastall} na sljedeći način:
% \lstset{language=Pascal} NACI NACINA ZA UBACIT BASH KOD
blastall -p blastp

i tako dalje...


SLIKA CJEVOVODA 
\label{fig:cjevovod}
(prikazuje podakte a ne filtre radi boljeg shvaćanja)

tekst o objetkima

SLIKA objekata (class dijagram)


\section{tax}

\section{server}
\label{sec:server}
    dio po dio

    jQuery

    komunikacija pipelinea i klijenta, log, dekoratori

automatizacija (posla kroz cjevovod->treba ići u neko uvodno poglavlje o namjeri)

robusnost
