\chapter{Teoretski uvod}
\label{chap:teoretski-uvod}

kljucna uloga proteina

srodnost / evolucija / otkrivanje

\section{Homologija proteina}

Homologija u biološkom smislu predstavlja slične osobine među vrstama na različitim razinama organizacije života, poput organa, tkiva, stnice ili molekule.
Homologne osobine uočene među jedinkama različitih vrsta obično upućuju na zajedničke pretke tih vrsta u evoluciji. Međutim, u molekularnoj biologiji termin homolog se često koristi i za naznačavanje sličnosti. bez obzira na genetsko srodstvo \cite{bioinfo1}

Za homologne sekvence proteina kažemo da su ortologne kad su direktni potomci neke sekvence u zajedničkom pretku, bez da su prošle duplikaciju gena. Drugim riječima, ortologne sekvence se mogu naći u jedinkama različitih vrsta, a obavljaju istu funkciju u svim tim vrstama. Paralogne sekvence su homologne sekvence koje su nastale od dvije različite kopije nekog gena koji je prošao kroz proces duplikacije gena u nekom zajedničkom evolucijskom pretku. Paralozi se mogu naći u jedinkama jedne ili više vrsta te obavljaju slične funkcije.

chart ortho-para


ideja...


