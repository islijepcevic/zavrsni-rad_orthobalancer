\chapter{Zaključak}
\label{chap:zakljucak}

Orthobalancer nudi jedinstven i izravan način odabira skupova ortolognih
proteinskih sekvenci iz bliskih taksonomskih grana. Aplikacija prima nekoliko
paralognih proteinskih sekvenci, a vraća skup ortologa za svaki paralog, uz
informaciju koliko je pojedini ortolog evolucijski udaljen od ortologa iz
skupova ostalih paraloga.

Aplikacija je izrađena kao mrežna aplikacija te je dostupna na URL-u
\footnotesize{\url{http://orthobalancer.bmad.bii-sg.org}}. \normalsize
Mrežna aplikacija izrađena je robusno, a unošenje podataka dinamički je
prilagodivo potrebama korisnika.

Aplikacija u pozadini radi na razini vrsta i taksonomske hijerarhije umjesto na
razini proteinskih sekvenci. Iako takav pristup nema garanciju da su pronađene
sekvence doista ortologne, prednost nad konvencionalnim metodama jest u tome što
je pokriven znatno veći broj proteina i vrsta. Klasične metode rade samo na
potpuno sekvencioniranim genomima kakvih je općenito malo, ograničeni su u
glavnom na kralježnjake, a sekvencioniranje cijelog genoma je sklono pogreškama.
Orthobalancer doprinosi kreiranju potrebnih balansiranih skupova sekvenci za
organizme za koje potpun genom nije poznat.

Za daljnje istraživanje, ideja zajedničkog podskupa vrsta među vrstama
usporedive taksonomske širine bi se mogla probati primijeniti za drugačije
namjene.

