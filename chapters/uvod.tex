\chapter{Uvod}
\label{chap:uvod}

Biološka sličnost među vrstama oduvijek je fascinirala čovječanstvo. Potpunije
razumijevanje sličnosti na molekularnoj razini moglo bi imati posljedice velikog
značenja. Na primjer, mnoge evolucijske veze među vrstama se mogu zaključiti na
temelju sličnosti, odnosno razlike genetskog materijala među vrstama.

Orthobalancer je mrežna aplikacija kojoj je cilj za zadani skup paralognih
proteina pronaći skup ortolognih proteina, odnosno proteina koji obavljaju
slične funkcije u drugim vrstama. Iako postoje mnoge metode za nalaženje
ortologa bazirane samo na genetskim informacijama, odnosno na DNA ili na
proteinskoj sekvenci, takve metode se mogu primijeniti samo na organizme čiji su
cjelokupni genomi sekvencirani u bazama podataka. Orthobalancer ovom problemu
pristupa na način da za svaki od ulaznih paralognih proteina prikupi skup vrsta
koje sadrže proteine dovoljno slične ulaznome proteinu, a zatim spustom kroz
taksonomsko stablo živog svijeta nalazi zajednički podskup vrsta kako bi odlučio
za koje proteine se može zaključiti da su ortolozi.

Dodatno, korisnik u aplikaciju može unijeti stupanj \emph{zamjenjivosti} na
način da određene čvorove taksonomskog stabla proglasi \emph{zamjenjivima}.
Tada, za dano podstablo, nije nužno da svi skupovi sadrže identičnu vrstu, već
je dozvoljeno odstupanje pri kojem se bilo koja vrsta mogu \emph{balansirati}
kao reprezentativne vrste svog skupa za dano \emph{zamjenjivo} podstablo.

Detaljniji biološki uvod dan je u poglavlju \ref{chap:teoretski-uvod}. Poglavlje
\ref{chap:podaci} daje pregled o korištenim formatima podataka, bazama te
ulaznim i izlaznim podacima. Kratki opis implementacije dan je u poglavlju
\ref{chap:implementacija}. Središnji dio aplikacije --- cjevovod --- opisan je u
poglavlju \ref{chap:cjevovod}. Modul za nalaženje i balansiranje vrsta objašnjen
je u poglavlju \ref{chap:tax}. Poglavlje \ref{chap:rezultati} komentira dobivene
rezultate. Zaključak je iznesen u poglavlju \ref{chap:zakljucak}.


 
