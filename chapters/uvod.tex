\chapter{Uvod}
\label{chap:uvod}

Komparativna analiza DNK \footnote{Deoksiribonukleinska kiselina} i proteinskih
sekvenci je disciplina koja na temelju proteinskih sekvenci i gena pokušava
pronaći srodne sekvence i primijeniti očuvanost sekvence i strukture proteina
kako bi predvidjela zajedničke biokemijske aktivnosti i biološke funkcije
proteina. \cite{koonin} Komparativnom analizom se direktno može uvidjeti koji su
dijelovi proteina pod \emph{evolucijskim pritiskom}, odnosno koliko često
mutiraju. Na osnovu tako prikupljenih podataka može se zaključiti koji su
dijelovi proteina od funkcionalne važnosti za protein, a time i za organizam.

Orthobalancer je mrežna aplikacija koja kreiranjem skupa relevantnih proteinskih
sekvenci pomaže komparativnoj analizi proteinskih sekvenci. Za zadani skup
paraloga pronalazi se skup ortologa, odnosno proteina koji obavljaju slične
funkcije u drugim vrstama. Iako postoje preciznije metode za nalaženje ortologa,
takve metode se mogu primijeniti samo na organizme čiji su cjelokupni genomi
sekvencirani u bazama podataka.  Orthobalancer ovom problemu pristupa na način
da za svaki od ulaznih paralognih proteina prikupi skup vrsta koje sadrže
proteine dovoljno slične ulaznome proteinu, a zatim spustom kroz taksonomsko
stablo živog svijeta nalazi zajednički podskup vrsta kako bi odlučio za koje
proteine se može zaključiti da su ortolozi.

Dodatno, korisnik u aplikaciju može unijeti stupanj \emph{zamjenjivosti} na
način da određene čvorove taksonomskog stabla proglasi \emph{zamjenjivima}.
Tada, za dano podstablo, nije nužno da svi skupovi sadrže identičnu vrstu, već
je dozvoljeno odstupanje pri kojem se bilo koja vrsta može \emph{balansirati}
kao reprezentativne vrste svog skupa za dano \emph{zamjenjivo} podstablo.

Detaljniji biološki uvod dan je u poglavlju \ref{chap:teoretski-uvod}. Poglavlje
\ref{chap:podaci} daje pregled o korištenim formatima podataka, bazama te
ulaznim i izlaznim podacima. Kratki opis implementacije dan je u poglavlju
\ref{chap:implementacija}. Središnji dio aplikacije --- cjevovod --- opisan je u
poglavlju \ref{chap:cjevovod}. Modul za nalaženje i balansiranje vrsta objašnjen
je u poglavlju \ref{chap:tax}. Poglavlje \ref{chap:rezultati} komentira dobivene
rezultate. Zaključak je iznesen u poglavlju \ref{chap:zakljucak}.


 
