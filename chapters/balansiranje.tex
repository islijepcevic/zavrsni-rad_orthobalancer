\chapter{Nalaženje zajedničkih vrsta}
\label{chap:tax}

% TODO pomaknut ovo u uvod negdje; sredit citate
Pronalazak zajedničkih vrsta najbitniji je modul Orthobalancera. U suštini, ovaj
modul nije logički vezan za nalaženje ortologa, no za potrebe Ortobalancera je
svojim sučeljima prilagođen njegovom cjevovodu. Nalaženje zajedničkih vrsta među
skupovima ulaznih vrsta predstavlja jednu novu dimenziju u pronalasku ortolognih
proteina. Konvencionalno traženja ortologa\cite{tatusov1997genomic,
bharatham2011determinants} jest kvalitetnije jer se pretraživanje odvija na
razini sekvence, ali je moguće samo za genome koji su u potpunosti istraženi i
zapisani u baze podataka.\cite{flicek2011ensembl} Podizanje potrage za
ortolozima na razinu vrsta omogućava da se ortolozi pronađu i među vrstama čiji
genomi nisu još zabilježeni.

Ovaj modul za svoj rad koristi podatke iz NCBI-jeve \emph{Taxonomy} baze.
Koristi se čvorovi taksonomskog stabla živog svijeta i znanstvena imena
pridijeljena čvorovima. Čvorovi su identificirani svojim jedinstvenim
identifikatorima \emph{tax\_id}, a svaki čvor ima pridruženo jedno ili više
korištenih imena. Samo jedno od tih imena je označeno kao znanstveno i također
je jedinstveno za svaki čvor, uz neke iznimke poput sintetički stvorenih vrsta
koje dijele znanstveno ime \emph{Synthetic construct}. Datoteke koje sadrže
podatke iz ovih baza su prilično velike te njihovo učitavanje, prilagođavanje i
korištenje najviše utječe na trajanje izvođenja programa.

Na početku rada modula svi podaci se prilagođavaju za potrebe modula. Iz
cjevovoda se od svake instance \emph{ProteinHolder}-a uzimaju zabilježene vrste,
a iz popisa zamjenjivih čvorova koje zadaje korisnik se prikupljaju svi zadani
čvorovi. Svim prikupljenim imenima se tada pridružuje \emph{tax\_id} iz baze.

Nakon toga se gradi taksonomsko stablo u memoriji. Stablo je izvedeno kao veliki
rječnik kojem za ključeve koristi \emph{tax\_id}, a svaki čvor je objekt razreda
\emph{Node}. Razred \emph{Node} sadrži sljedeće bitne podatke: \emph{tax\_id}
roditeljskog čvora, listu djece, brojač za svaku grupu ortologa, ukupni brojač
grupa, listu zamjenskih roditeljskih čvorova te zastavicu je li balansiran.
Brojači se inicijaliziraju na nulu, a kasnije tijekom algoritma će koristiti kao
broj vrsta pojedine grupe ortologa koje se nalaze pod danim čvorom. Ukupni
brojač grupa govori koliko ortolognih grupa ima svoga predstavnika pod nekim
čvorom. Lista zamjenskih roditeljskih čvorova se inicijalizira na praznu listu,
a bit će popunjena svim čvorovima koji su od korisnika označeni kao zamjenski te
se nalaze iznad trenutno razmatranog čvora. Zastavica za balansiranje koristi se
kasnije tijekom postupka balansiranja podstabala ispod zamjenskih čvorova.
Algoritam je opisan u nastavku, a dan je i njegov pseudokod \ref{algo:balance}.

Sljedeći koraci predstavljaju inicijalizaciju stabla. Najprije se za svaku
ortolognu grupu prolazi kroz sve vrste. Za svaku vrstu pronalazi se njen čvor,
odnosno list u stablu. Brojač tog lista za trenutnu ortolognu grupu se
inicijalizira na $1$. Istovremeno se ukupni brojač grupa u tome listu postavi na
$1$ te se pozove metoda koja propagira ukupni brojač grupa sve do korijena
stabla, pazeći da ukupni brojač niti u jednom čvoru ne bude više od jednom
povećan za neku grupu ortologa.
% TODO možda napisati pseudokod

Nakon toga treba inicijalizirati zamjenske čvorove i pripadna podstabla. Za
svaki zamjenski čvor poziva se rekurzivna funkcija koja se spušta do listova i
svakome čvoru u njegovu listu zamjenskih roditeljskih čvorova dodaje
\emph{tax\_id} zamjenskog čvora nad kojim je rekurzija pozvana.  Time svaki čvor
sadrži informaciju kojim zamjenskim podstablima pripada.

\begin{algorithm}[h!]

\centering
\caption{Obilazak stabla --- balansiranje vrsta}
\label{algo:balance}

\begin{algorithmic}

%TODO - pozvano s cvorom 'root'
\State $ULAZ:$ taksonomsko stablo, skup zamjenjivih čvorova, $n$ skupova vrsta
\State $SREDIŠNJE STRUKTURE PODATAKA:$ %TODO (sve postavljeno - opisati)
\State $IZLAZ:$ višeslojni rječnik $out$koji za svaki balansirani čvor sadrži
listu vrsta predstavnica iz svake grupe $g$  

\Function{obidiStablo}{$cvor, grupe, zamjenskiCvorovi$}
    \If{$čvor.zamjenski()$}
    \Comment{metoda vraća istinu ako je čvor u zamjenskom podstablu} \\
        \Return $balansiraj(cvor, grupe, zamjenskiCvorovi)$
    \Else
        \If{$cvor.list()$}
        \Comment{metoda vraća istinu ako je čvor list}
            \ForAll{$grupa \in grupe$}
                \State $izlaz[grupa]['nezamjenski'] \gets cvor.ime$
            \EndFor
        \Else
            \State $izlaz \gets$ new $izlaz$
            \ForAll{$dijete \in cvor.dijeca$}
                \If{$dijete.ukupniBrojac = čvor.ukupniBrojac$}
                    \State $izlaz \in izlaz \cup obidiStablo(dijete, grupe,
zamjenskiCvorovi)$
                \EndIf
            \EndFor
        \EndIf
    \EndIf
    \Return $izlaz$
\EndFunction

\end{algorithmic}
\end{algorithm}  


Slijedi algoritam \ref{algo:balance}. Algoritam rekurzivno prolazi stablom,
polazeći od listova. U zamjenskim podstablima balansira grupe ortologa na
sljedeći način. Nakon što se obiđu listovi, svaki čvor, za svaku od grupa
ortologa, postavlja svoj brojač za tu grupu na vrijednost zbroja brojača svoje
djece, pri čemu se od djece ne razmatraju već balansirani čvorovi. U slučaju da
nijedan brojač u čvoru ne ostane na vrijednosti $0$, čvor se proglašava
balansiran. Drugim riječima, u svakoj grupi ortologa postoji barem jedna vrsta
ispod toga čvora te se one mogu ispisati kao uparene ili zamjenjive vrste.  U
idealnom, ali i najčešćem slučaju, desit će se da je vrsta prisutna u svim
grupama ortologa te će tada sam čvor promatrane vrste biti proglašen
balansiranim i roditelji ga zbog oga neće razmatrati. Za primjenu nalaženja
ortolognih proteina to bi značilo da neka vrsta, na primjer Homo sapiens, sadrži
ortologni protein svakome od ulaznih paraloga. Algoritam je tako osmišljen kako
bi se balansirale što srodnije vrste.  Što se tiče dijelova stabla koji nisu pod
niti jednim zamjenskim čvorom, algoritam ne provodi postupak balansiranja te se
u izlaz dodaju jedino one vrste koje su prisutne u svim grupama ortologa. Izlaz
ovog algoritma vraća se pozivatelju, odnosno cjevovodu.


